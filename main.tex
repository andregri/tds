\documentclass{report}
\usepackage[a4paper, total={6in, 9in}]{geometry}
\usepackage{tikz}
\usepackage{amsmath}
\usepackage{pgfplots}
\usepackage{caption}
\usepackage{subcaption}
\usepackage{hyperref}
\usepackage{stackrel}
\usepackage{array}
\usepackage{amssymb}
\usepackage{verbatim}
\usepackage{wrapfig}

\hypersetup{
	colorlinks=true,
	linkcolor=blue,
	filecolor=magenta,      
	urlcolor=cyan,
	bookmarks=true,
}

\graphicspath{{img/}{../img/}}

\usepackage{subfiles}

\begin{document}
	%user defined commands
	%real numbers
\newcommand{\R}{\rm I\!R}

%complex number
\newcommand{\C}{\mathbb{C}}

\newcommand{\compl}[2]{#1 + j #2}
\newcommand{\ncompl}[2]{#1 - j #2}

\newcommand{\re}[1]{\Re \left\lbrace #1 \right\rbrace }
\newcommand{\im}[1]{\Im \left\lbrace #1 \right\rbrace }

%derivate d#2/d#1
%#1 function
%#2 variable of derivation
\newcommand{\der}[2]{\frac{\mathrm{d} #1}{\mathrm{d} #2}}

%partial derivate 
\newcommand{\pder}[2]{\frac{\partial #1}{\partial #2}}

%polynom degree
\newcommand{\pDeg}[1]{\partial eg \left\lbrace  #1 \right\rbrace }

%2x2 matrix
\newcommand{\matTwo}[4]{\left[ \begin{array}{cc} #1 & #2\\ #3 & #4 \end{array} \right]}
	\newcommand{\Lt}{\mathcal{L}}
\newcommand{\aLt}{\Lt^{-1}}

% L{#1}
\newcommand{\Lbrace}[1]{\Lt \left\lbrace #1 \right\rbrace}

% antitrasformata Laplace
\newcommand{\aLbrace}[1]{\aLt \left\lbrace #1 \right\rbrace}

%definizione laplace (integrale)
\newcommand{\LaplDef}[3]{\int_{0^{-}}^{+\infty} #1 e^{- #2 #3} \mathrm{d} #3}

%trasformata di laplace: 
% #1=funzione partenza
% #2=funzione trasformata
% #1 arrow #2
\newcommand{\Lapl}[2]{#1 \quad \xrightarrow{\Lt} \quad #2}

%antitrasformata di laplace
\newcommand{\aLapl}[2]{#1 \quad \xrightarrow{\aLt} \quad #2}
	%contents
	\tableofcontents
	\pagebreak
	\chapter{Introduzione}
	\subfile{section/introduzione}	
	\pagebreak
	\chapter{Funzioni generalizzate}
	\subfile{section/funzioni_generalizzate}
	\pagebreak
	\chapter{Trasformata di Laplace}
	\subfile{section/trasformata_laplace}
	\pagebreak
	\chapter{Anti-Trasformata di Laplace}
	\subfile{section/anti_trasformata_laplace}
	\pagebreak
	\chapter{Sistemi lineari}
	\subfile{section/applicazione_laplace}
	\subfile{section/sistemi_lineari}
	\subfile{section/stabilita}
	\subfile{section/rappresentazione_sistema}
	\subfile{section/risposta_gradino_ordine_1_2}
	\pagebreak
	\chapter{Algebra dei blocchi}
	\subfile{section/connessioni}
	\subfile{section/risposta_forzata_sistemi_interconnessi}
	\pagebreak
	\chapter{Equazione di stato}
	\subfile{section/equazione_stato}
	\subfile{section/punti_equilibrio}
	\pagebreak
	\chapter{Riepilogo}
	\subfile{section/riepilogo_trasf_laplace}
\end{document}