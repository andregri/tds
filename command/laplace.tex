\newcommand{\Lt}{\mathcal{L}}
\newcommand{\aLt}{\Lt^{-1}}

% L{#1}
\newcommand{\Lbrace}[1]{\Lt \left\lbrace #1 \right\rbrace}

% antitrasformata Laplace
\newcommand{\aLbrace}[1]{\aLt \left\lbrace #1 \right\rbrace}

%definizione laplace (integrale)
\newcommand{\LaplDef}[3]{\int_{0^{-}}^{+\infty} #1 e^{- #2 #3} \mathrm{d} #3}

%trasformata di laplace: 
% #1=funzione partenza
% #2=funzione trasformata
% #1 arrow #2
\newcommand{\Lapl}[2]{#1 \quad \xrightarrow{\Lt} \quad #2}

%antitrasformata di laplace
\newcommand{\aLapl}[2]{#1 \quad \xrightarrow{\aLt} \quad #2}