\documentclass[../main.tex]{subfiles}

\begin{document}
	\section{Equazione di stato}
		Abbiamo visto che i sistemi MIMO possono essere rappresentati attraverso un sistema di equazioni differenziali o in forma matriciale.\\
		\begin{center}
			\begin{minipage}{.9\linewidth}
				L'equazione di stato \'e una rappresentazione dei sistemi dinamici (non necessariamente lineare) che mette in relazione ingressi e uscite sfruttando una quantit\'a (interna al sistema) detta \textit{stato del sistema}.\\
			\end{minipage}
		\end{center}
		Lo stato $ \underline{x}(t) $ rappresenta una \textit{foto istantanea} di $ S $ e deve descriverne completamente la situazione. Per determinare $ \underline{x}(t) $ per $ t \geq t_0 $ saranno sufficienti $ \underline{x}(t^-_0) $ e $ \underline u(\tau) $ per $ \tau \geq t_0 $.\\
		\begin{align*}
			\underline{y}(t) = \left[ \begin{array}{c}y_1(t)\\\cdots\\y_{n_y}(t)\end{array} \right]
			&&\underline{u}(t) = \left[ \begin{array}{c}u_1(t)\\\cdots\\u_{n_u}(t)\end{array} \right]
			&&\underline{x}(t) = \left[ \begin{array}{c}x_1(t)\\\cdots\\y_{n_x}(t)\end{array} \right]\\
			\medskip
			n_y = dim(\underline{y}(t)) &&n_u = dim(\underline{u}(t)) &&n_x = dim(\underline{x}(t))
		\end{align*}
		Le equazioni per descrivere il sistema in notazione matriciale saranno:
		\begin{align*}
			\underline{\dot{x}}(t) &= A \underline{x}(t) + B \underline{u}(t)\\
			\underline{y}(t) &= C \underline{x}(t) + D \underline{u}(t)
		\end{align*}
		dove le dimensioni delle matrici sono:
		\[ A \in \R^{n_x x\ n_x} \qquad B \in \R^{n_x x\ n_u} \qquad C \in \R^{n_y x\ n_x} \qquad D \in \R^{n_y x\ n_u} \]
		Si nota dalla seconda equazione che il vettore delle uscite dipende istantaneamente dalla stato $ (\underline x) $ e dal controllo $ (\underline u) $.\\
		Dalla prima equazione invece vediamo che lo stato dipende dinamicamente dallo stato stesso e dal controllo.\\
		\linebreak 
		Scrivendo esplicitamente le due equazioni otteniamo:
		\newcommand{\eq}[2]{#2_{11} #1_1(t) + #2_{12} #1_2(t) + \cdots + #2_{1n_#1} #1_{n_#1}(t)}
		\begin{align*}
			\dot x_1(t) &= \eq{x}{a} && + \eq{u}{b}\\
			&\colon\\
			\dot x_{n_x}(t) &= \eq{x}{a} && + \eq{u}{b}\\
			\smallskip\\
			y_1(t) &= \eq{x}{c} && + \eq{u}{d}\\
			&\colon\\
			y_{n_y}(t) &= \eq{x}{c} && + \eq{u}{d}\\
		\end{align*}
		
	\section{Soluzione dell'equazione di stato}
		Significa trovare $ \underline x(t) $ e di conseguenza $ \underline y(t) $, dati $ \underline x(t_0^-) $ e $ u(\tau) $ per $ \tau \in [0, t_0] $.\\
		Si dimostra che:
		\[ \underline x(t) = C e^{At} \underline x(0^-) + C \int_{0^-}^{t} e^{A(t-\tau)} B \underline u(\tau) d\tau + D \underline u(t) \]
		dove:
		\begin{itemize}
			\item 
				$ e^{At} \underline x(0^-) \quad $ evoluzione libera dello stato.\\
				\'E l'uscita che dipende solo dalle $ n_x $ condizioni iniziali, pertanto si chiama vettore delle uscite libere.
			\item
				$ \int_{0^-}^{t} e^{A(t-\tau)} B \underline u(\tau) d\tau \quad $ integrale di convoluzione che rappresenta l'evoluzione forzata dello stato.
		\end{itemize}
	
	\subsection{Come si calcola $ e^{At} $}
		\[ e^{At} = I + \frac{At}{1!} + \frac{(At)^2}{2!} + \frac{(At)^3}{3!} + \dots = \sum_{k=0}^{k=\infty} \frac{A^k t^k}{k!} \]
		Per effettuare questo calcolo distinguiamo vari casi:
		\begin{itemize}
			\item 
				$ A^k = 0\ \text{per}\ k \geq \bar k $ (matrice nil-potente di ordine $ \bar k $): usare la formula esplicitamente.\\
				\subsubsection*{Esempio}
					\[ A = \left[ \begin{array}{cc} 0 & 1\\ 0 & 0 \end{array} \right] \qquad
					   A^2 = \left[ \begin{array}{cc} 0 & 0\\ 0 & 0 \end{array} \right] \qquad
					   A^k = \left[ \begin{array}{cc} 0 & 1\\ 0 & 0 \end{array} \right] \]
					Abbiamo trovato che A \'e una matrice nil-potente di ordine 2.
					\[ e^{At} = I + At = \left[ \begin{array}{cc} 1 & 0\\ 0 & 1 \end{array} \right] + \left[ \begin{array}{cc} 0 & t\\ 0 & 0 \end{array} \right] = \left[ \begin{array}{cc} 1 & t\\ 0 & 1 \end{array} \right] \]
			\item
				A \'e una matrice diagonale:
				\[ e^{At} = \left[ \begin{array}{ccc} \frac{\lambda_1^k t^k}{k!} & \dots & 0\\ \vdots & \ddots & \vdots\\ 0 & \dots & \frac{\lambda_{n_x}^k t^k}{k!} \end{array} \right] = 
				\left[ \begin{array}{ccc} e^{\lambda_1 t} & \dots & 0\\ \vdots & \ddots & \vdots\\ 0 & \dots & e^{\lambda_{n_x} t} \end{array} \right] \]
			\item
				Negli altri casi usiamo questa formula che verr\'a chiarita in seguito:
				\[ e^{At} = \Lbrace{(sI-A)^{-1}} \]
				\subsubsection*{Esempio}
					Ripetiamo il calcolo dell'esercizio precedente con $ A = \matTwo{0}{1}{0}{0} $
					\[ e^{At} = \aLbrace{(sI-A)^{-1}} = \aLbrace{\matTwo{\frac{1}{s}}{\frac{1}{s^2}}{0}{\frac{1}{s}}} = \matTwo{1}{t}{0}{1} 1(t) \]
		\end{itemize}
	
	\subsection{Polinomio caratteristico}
		Data una matrice quadrata A, il polinomio caratteristico \'e definito come:
		\[ \phi(s) = det(sI - A) \]
		Inoltre si pu\'o calcolare come il mcm di tutti i denominatori diversi da zero di tutti i determinanti diversi da zero di tutte le sottomatrici quadrate di $ (sI-A)^{-1} $. Questo metodo pu\'o essere utile nel caso in cui ci venga fornita direttamente $ (sI-A)^{-1} $.
	\subsubsection*{Esempio}
		\[ A = \matTwo{1}{0}{0}{2} \]
		\[ \phi(s) = det \left( s \matTwo{1}{0}{0}{1} - \matTwo{1}{0}{0}{2} \right) = det \left( \matTwo{s-1}{0}{0}{s-2} \right) = (s-1)(s-2) \]
		Gli autovalori sono: $ s_1 = 1 \quad s_2 = 2 $ entrambi con molteplicit\'a 1.
	\subsubsection*{Esempio}
		Usiamo il secondo metodo per calcolare il polinomio caratteristico data:
		\[
			(sI-A)^{-1} =
			\begin{bmatrix}
				\frac{1}{s} & 0\\
				0 & \frac{1}{s}
			\end{bmatrix}
		\]
		Il suo determinante \'e: $ det (sI-A)^{-1} = \frac{1}{s^2} $.\\
		Ci sono 5 sottomatrici: 4 sono i singoli elementi, il quinto \'e la matrice stessa:
		\[ \phi(s) = mcm(s, s, s^2) = s^2 \]
	\subsection*{Propriet\'a}
		$ \phi(s) $ \'e un polinomio annullante per A se $ \phi(A) = [0] $ (matrice nulla).\\
		Il suo grado \'e: $ \pDeg{\phi(s)} = n_x $.
	\subsubsection*{Esempio}
		Consideriamo la matrice e il polinomio caratteristico dell'esempio precedente:
		\[ \phi(s) = (s-1)(s-2) = s^2 - 3s + 2 \]
		\[ \phi(A) = A^2 + -3A + 2I = \matTwo{1}{3}{0}{2} \matTwo{1}{3}{0}{2} - \matTwo{3}{9}{0}{6} + \matTwo{2}{0}{0}{2} = \matTwo{0}{0}{0}{0}\]
		
	\subsection{Polinomio minimo}
		Il polinomio minimo $ m(s) $ \'e il polinomio annullante, monico e di grado pi\'u basso per A.
	\subsection*{Propriet\'a}
		\[ m(s) \subseteq \cdot\ \phi(s) \]
		$ \subseteq \cdot $ significa che tutte le radici di $ \phi(s) $ sono anche radici di $ m(s) $, eventualmente con molteplicit\'a minore.\\
		\linebreak
		$ m(s) $ \'e calcolabile come il mcm di tutti i denominatori dei termini non nulli della matrice $ (sI - A)^{-1} $.\\
		Il suo grado \'e: $ \pDeg{m(s)} \leq n_x $.\\
	\subsubsection*{Esempio}
		Per fare l'inversa di una matrice diagonale, bisogna invertire i suoi elementi. 
		\[ A = \matTwo{0}{0}{0}{0} \qquad (sI - A)^{-1} = \matTwo{s}{0}{0}{s}^{-1} = \matTwo{\frac{1}{s}}{0}{0}{\frac{1}{s}} \]
		\[ m(s) = mcm( s, s) = s \]
	\subsubsection*{Esempio}
		Per fare l'inversa di una matrice triangolare, gli elementi sulla diagonale si invertono, gli altri bisogna calcolarli.
		\[ A = \matTwo{0}{1}{0}{0} \qquad (sI - A)^{-1} = \matTwo{s}{-1}{0}{s}^{-1} = \matTwo{\frac{1}{s}}{\frac{1}{s^2}}{0}{\frac{1}{s}} \]
		\[ m(s) = mcm( s, s^2, s) = s^2 \]
	\subsubsection*{Esempio}
		\[
			(sI-A)^{-1} =
			\begin{bmatrix}
				\frac{1}{s} & \frac{1}{s^2}\\
				\frac{1}{s+1} & \frac{1}{s}
			\end{bmatrix}
		\]
		\[ m(s) = mcm \left\lbrace s, s+1, s, s^2 \right\rbrace = s^2(s+1) \]
		Ma questo risultato \'e impossibile perch\'e $ \pDeg{m(s)} = 3 $ mentre la matrice \'e $ 2 \times 2 $. Questo significa che la matrice data non pu\'o derivare da un sistema lineare, quindi non ha senso svolgere ulteriori calcoli si essa.
		
	\section{Trasformata della soluzione dell'equazione di stato}
		Calcolo la trasformata di $ \dot{\underline x}(t) = A \underline x(t) + B \underline u(t) $ con la propriet\'a della derivata:
		\[ s \underline X(s) - \underline x(0^-) = A \underline X(s) + B \underline U(s) \]
		\[ s \underline X(s) - A \underline X(s) = \underline x(0^-) + B \underline U(s) \]
		\[ (sI - A) \underline X(s) = \underline x(0^-) + B \underline U(s) \]
		\begin{equation}
			X(s) = (sI-A)^{-1} \underline x(0^-) + (sI-A)^{-1} B \underline U(s)
		\end{equation}
		Analogamente la trasformata della seconda equazione $ \underline y(t) = C \underline x(t) + D \underline u(t) $:
		\begin{equation}
			\underline Y(s) = C (sI-A)^{-1} \underline x(0^-) + [C(sI-A)^{-1}B + D]\ \underline U(s)
		\end{equation}
		Da quest'ultima ricaviamo la funzione di trasferimento:
		\[ T(s) = C (sI-A)^{-1} B + D \]
		Le matrici associate alle varie risposte con i rispettivi polinomi caratteristici e minimi:
		\begin{align*}
			&X_l\ (n_x\ \times\ n_x):\ (sI-A)^{-1} &&\longrightarrow &m(s), \phi(s)\\
			&X_f\ (n_x\ \times\ n_u):\ (sI-A)^{-1}B &&\longrightarrow &m_c(s), \phi_c(s)\\
			&Y_l\ (n_y\ \times\ n_x):\ C(sI-A)^{-1} &&\longrightarrow &m_o(s), \phi_o(s)\\
			&Y_f\ (n_y\ \times\ n_u):\ C(sI-A)^{-1}B &&\longrightarrow &m_{co}(s), \phi_{co}(s)\\
		\end{align*}
		Le ultime tre matrici sono combinazioni lineari della prima $ (sI-A)^{-1} $, quindi a denominatore non potranno comparire ulteriori termini. Allora per i polinomi si ha:
		\[ \begin{array}{ccc}
						 &\subseteq \phi_c(s) &\\
			\phi_{co}(s) & 					  &\subseteq \phi(s)\\
						 &\subseteq \phi_o(s) &\\
		\end{array} \qquad
		\begin{array}{ccc}
					  &\subseteq m_c(s) &\\
			m_{co}(s) & 				&\subseteq m(s)\\
					  &\subseteq m_o(s) &\\
		\end{array} \]
		
	\subsection*{Esempio}
		Dato il sistema:
		\[\begin{cases}
			\dot x = \matTwo{0}{1}{0}{0} x + \left[ \begin{matrix} 0\\1 \end{matrix} \right]\\
			\smallskip\\
			y = \left[ \begin{matrix} 1 &1 \end{matrix} \right] x
		\end{cases} \]
		Calcolare $ x(t) $ e $ y(t) $ quando $ x(0^-) = \left[ \begin{matrix} 1\\ 2 \end{matrix} \right] $ e $ u(t) = 1(t) $.
		Calcoliamo le trasformate delle soluzioni:
		\begin{align*}
			X(s) &= (sI-A)^{-1} x(0^-) + (sI-A)^{-1} B\ U(s) =\\
			&= \matTwo{\frac{1}{s}}{\frac{1}{s^2}}{0}{\frac{1}{s}} \left[ \begin{matrix} 1\\ 2 \end{matrix} \right] + \matTwo{\frac{1}{s}}{\frac{1}{s^2}}{0}{\frac{1}{s}} \left[ \begin{matrix} 0\\1 \end{matrix} \right] \frac{1}{s} = \left[ \begin{matrix} \frac{1}{s}+\frac{2}{s^2}+\frac{1}{s^3}\\ 2\frac{1}{s}+\frac{1}{s^2} \end{matrix} \right] 
		\end{align*}
		\[ x(t) = \left[ \begin{matrix} 1+2t+\frac{1}{2}t^2\\ 2+t \end{matrix} \right] 1(t) \]
		\[ y(t) = \left[ \begin{matrix} 1 &1 \end{matrix} \right] x(t) = (3+3t+\frac{1}{2}t^2)\ 1(t) \]
\end{document}