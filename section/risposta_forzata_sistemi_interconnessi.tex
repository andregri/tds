\documentclass[../main.tex]{subfiles}

\begin{document}
	\newcommand{\T}[1]{T^{(#1)}(s)}
	%
	\section{Algebra dei blocchi}
	L'algebra dei blocchi permette di calcolare la funzione di trasferimento $ T(s) $ totale di un sistema costituito dall'interconnessione di pi\'u sistemi.\\
	Solitamente la risposta forzata di un generico sistema MIMO \'e: $ \underline{y} = T \underline{u} $.\\
	L'algebra dei blocchi consiste nel considerare tutte le variabili in gioco $ (\underline{y}, T, \underline{u}) $ come scalari e operare su di essi come se fossero quantit\'a scalari. Grazie a passaggi algebrici classici, quindi si pu\'o ottenere facilmente la funzione di trasferimento totale. 
	\section{Risposta forzata di sistemi interconnessi}
	La connessione tra pi\'u sistemi produce un nuovo sistema caratterizzato da:
	\begin{itemize}
		\item ingressi: $ \quad \underline{u}(t) = \left[ \begin{array}{c}u_1(t)\\\cdots\\u_{n_u}(t)\end{array} \right] $
		\item uscite: $ \quad \underline{y}(t) = \left[ \begin{array}{c}y_1(t)\\\cdots\\y_{n_y}(t)\end{array} \right] $
		\item $ T_{ij}(s) = T_{y_i u_j}(s) \quad $ funzione di trasferimento tra l'ingresso $ u_j $ e l'uscita $ y_i $
	\end{itemize}
	Quindi un sistema MIMO non sar\'a caratterizzato da una sola funzione di trasferimento, ma da una \textbf{matrice di trasferimento}:
	\[ T(s) = \left[ \begin{array}{cccc}
	T_{11} & T_{12} & \dots & T_{1n_u} \\ 
	T_{21} & T_{22} & \cdots & \colon \\ 
	\colon & \colon & \ddots & \colon \\ 
	T_{n_y1}& T_{n_y2} & \cdots & T_{n_yn_u} 
	\end{array} \right] \]
	Questa matrice ha:
	\begin{itemize}
		\item tante colonne quante gli ingressi.
		\item tante righe quante le uscite.
	\end{itemize}
	Quindi la risposta forzata del sistema si ottiene moltiplicando la matrice di trasferimento con il vettore colonna degli ingressi:
	\[ \left[ \begin{array}{c}y_1(t)\\\cdots\\y_{n_y}(t)\end{array} \right] = \left[ \begin{array}{cccc}
	T_{11} & T_{12} & \dots & T_{1n_u} \\ 
	T_{21} & T_{22} & \cdots & \colon \\ 
	\colon & \colon & \ddots & \colon \\ 
	T_{n_y1}& T_{n_y2} & \cdots & T_{n_yn_u} 
	\end{array} \right] \left[ \begin{array}{c}u_1(t)\\\cdots\\\cdots\\u_{n_u}(t)\end{array} \right] \]
	%
	\subsection{Stabilit\'a BIBO}
	\subsubsection*{Definizione}
	Un sistema \'e stabile BIBO \textbf{se e solo se} qualsiasi ingresso limitato, allora l'uscita forzata $ y_f(t) $ \'e limitata.\\
	$ y_f(t) $ \'e un vettore che ha come elementi $ y_{f,1}(t), y_{f,2}(t), \dots, y_{f,n_y}(t) $, quindi affinch\'e sia stabile, tutte le componenti devono essere limitate.
	\subsubsection*{Condizioni}
	Nessuno elemento della matrice di trasferimento $ T(s) $, deve avere poli (radici del denominatore) a $ \Re \geq 0 $.
	\paragraph{Esempio}
	\[ T(s) = \matTwo{\frac{1}{s+1}}{\frac{1}{s}}{0}{\frac{1}{(s+1)^2}} \]
	Non \'e stabile BIBO perch\'e la componente $ T_{12}(s) $ ha un polo a $ \Re = 0 $. Infatti se poniamo un ingresso limitato: $ \underline u(t) = \left[ \begin{matrix} 0\\ 1(t) \end{matrix} \right] $ otteniamo un'uscita illimitata:
	\[ Y_f(s) = \matTwo{\frac{1}{s+1}}{\frac{1}{s}}{0}{\frac{1}{(s+1)^2}} \left[ \begin{matrix} 0\\ \frac{1}{s} \end{matrix} \right] = \left[ \begin{matrix} \frac{1}{s^2}\\ \frac{1}{s(s+1)^2} \end{matrix} \right] \]
	\begin{itemize}
		\item $ \frac{1}{s^2} $ nel tempo corrisponde a una rampa (illimitata);
		\item $ \frac{1}{s(s+1)^2} $ nel tempo corrisponde a un gradino e a un esponenziale decrescente (limitata).
	\end{itemize}
	%
	\section{Interconnessione di sistemi SISO}
	Dati due sistemi:
	\begin{itemize}
		\item $ S^{(1)} $: $ \quad Y_f^{(1)}(s) = T^{(1)}(s)U^{(1)}(s) $
		\item $ S^{(2)} $: $ \quad Y_f^{(2)}(s) = T^{(2)}(s)U^{(2)}(s) $
	\end{itemize}
	%
	\subsection{Serie}
	\begin{wrapfigure}{R}{.5\linewidth}%
		\centering
		\includegraphics[width=\linewidth]{interconnections/serie}%
	\end{wrapfigure}
	\leavevmode%
	\[ Y_f(s) = T^{(2)}(s)U^{(2)}(s) = T^{(2)}(s) T^{(1)}(s) U(s) \]
	La funzione di trasferimento del sistema complessivo \'e:
	\[ T(s) = T^{(2)}(s) T^{(1)}(s) \]
	(NB: l'ordine di questo prodotto vale anche per i sistemi MIMO che sono descritti con matrici)
	%
	\subsection{Parallelo}
	\[ Y_f(s) = T^{(1)}(s)U^{(1)}(s) + T^{(2)}(s)U^{(2)}(s) = \left[ T^{(1)}(s) + T^{(2)}(s) \right] \]
	\[ T(s) = T^{(1)}(s) + T^{(2)}(s) \]
	%
	\subsection{Retroazione (feedback)}
	\[ y= \T{1}e = \T{1}(u-v) = \T{1}[u-\T{2}y] = \T{1}u - \T{1}\T{2}y \]
	\[ [1+\T{1}\T{2}]\ y = \T{1}u \]
	\[ \begin{cases}
		\T{1} \quad \text{funzione di trasferimento in catena diretta}\\
		\T{2} \quad \text{funzione di trasferimento in catena inversa}
	\end{cases} \]
	In generale sia che $ v $ venga sottratto o sommato:
	\[ T(s) = \frac{\T{1}}{1\pm\T{1}\T{2}} \]
	\subsubsection{Loop algebrici}
	Se tutte le funzioni di trasferimento sono semplicemente proprie, allora si forma un \textbf{loop algebrico}, che rende irrealizzabile la retroazione.
	\paragraph{Esempio}
	\begin{wrapfigure}{R}{.5\linewidth}%
		\centering
		\includegraphics[width=\linewidth]{interconnections/loop_algebrico}%
	\end{wrapfigure}
	\leavevmode%
	Entrambe le funzioni di trasferimento sono amplificatori e $ u(t) = 0 $, $ e(t) =1 $.\\
	Sostituendo i valori e percorrendo la retroazione troveremo che $ e $ dovrebbe valere contemporaneamente $ 1 $ e $ -8 $.\\
	Infatti $ y $ dipende istantaneamente da $ e $, $ v $ d. i. da $ y $, $ e $ d. i. da $ v $, quindi otteniamo che $ e $ dipende istantaneamente da s\'e stessa, ovvero c'\'e un loop algebrico.
	%
	\pagebreak
	\subsubsection*{Esempio}
	\begin{wrapfigure}{R}{.5\linewidth}%
		\centering
		\includegraphics[width=\linewidth]{interconnections/esempio_retroazione}%
	\end{wrapfigure}
	\leavevmode%
	\'E possibile trovare $ k $ per avere un sistema stabile BIBO?\\
	Notiamo subito che non ci sono loop algebrici perch\'e non tutte le funzioni di trasferimento sono semplicemente proprie. Pertanto la funzione di trasferimento \'e:
	\[ T(s) = \frac{\frac{1}{s^2+1}}{1+\frac{k}{s^2+1}} = \frac{\frac{1}{s^2+1}}{\frac{s^2+1+k}{s^2+1}} = \frac{1}{s^2+1+k} \]
	\[ s^2 = -(k+1) \]
	Affinch\'e il sistema sia stabile BIBO, non ci devono essere radici a $ \Re \geq 0 $
	\begin{itemize}
		\item $ k<-1 \quad \Rightarrow \quad s_{1,2} = \pm \sqrt{-(k+1)} $
		\item $ k>-1 \quad \Rightarrow \quad s_{1,2} = \pm j \sqrt{k+1} $
		\item $ k=-1 \quad \Rightarrow \quad s_{1,2} = 0 $
	\end{itemize}
	Non esiste alcun $ k $ per cui il sistema sia stabile BIBO.
	%
	%esercizi
	\pagebreak
	\documentclass[../main.tex]{subfiles}

\begin{document}
	\section{Esercizio}
	Calcolare tutte le possibili funzioni di trasferimento del sistema:
	\begin{figure}[h!]
		\centering
		\begin{subfigure}{0.8\textwidth}
			\includegraphics[width=\textwidth]{interconnections/algebra_blocchi_es1_1}
		\end{subfigure}%
	\end{figure}
	\begin{itemize}
		\item $ T_{y_1 u_1}(s) $: poniamo $ u_2 = 0 $.\\
		\parbox[t]{\dimexpr\textwidth-\leftmargin}{%
			\vspace{-2.5mm}
			\begin{wrapfigure}{r}{0.5\textwidth}
				\centering
				\vspace{-\baselineskip}
				\includegraphics[width=\linewidth]{interconnections/algebra_blocchi_es1_2}
			\end{wrapfigure}
			Serie tra $ T^{(1)} $ e $ T^{(2)} $:
			\[ T_{serie} = T^{(1)} T^{(2)} = \frac{s+1}{s-1} \frac{1}{s+1} = \frac{1}{s-1} \]
			Retroazione tra $ T_{serie} $ e $ T^{(3)} $:
			\[ T_{y_1 u_1}(s) = \frac{T_{serie}}{1+T_{serie} T^{(3)}} = \frac{\frac{1}{s-1}}{1+\frac{1}{(s+1)(s-1)}} = \frac{\frac{1}{s-1}}{\frac{1}{(s^2-1+1)(s-1)}} = \frac{s+1}{s^2} \]
			Questa funzione di trasferimento \'e instabile BIBO perch\'e c'\'e un polo a $ \Re = 0 $.
		}
		%
		\item $ T_{y_1 u_2}(s) $: poniamo $ u_1 = 0 $.\\
		\parbox[t]{\dimexpr\textwidth-\leftmargin}{%
			\vspace{-2.5mm}
			\begin{wrapfigure}[10]{r}{0.5\textwidth}
				\centering
				\vspace{-\baselineskip}
				\includegraphics[width=\linewidth]{interconnections/algebra_blocchi_es1_3}
			\end{wrapfigure}
			Calcoliamo la serie nella catena inversa di retroazione:
			\[ T_{serie}(s) = \frac{-1}{(s+1)^2} \]
			Retroazione:
			\[ T_{y_1 u_2}(s) = \frac{\frac{s+1}{s-1}}{1-\frac{s+1}{s-1}\frac{-1}{(s+1)^2}} = \frac{\frac{s+1}{s-1}}{1+\frac{1}{(s+1)(s-1)}} = \frac{\frac{s+1}{s-1}}{\frac{s^2}{(s+1)(s-1)}} = \frac{(s+1)^2}{s^2} \]
			Questa funzione di trasferimento \'e instabile BIBO perch\'e c'\'e un polo a $ \Re = 0 $.
			Abbiamo ottenuto una funzione di trasferimento semplicemente propria perch\'e tra $ u_2 $ e $ y_1 $ in catena diretta c'\'e una dipendenza istantanea (dato dalla costante):
			\[ T^{(2)}(s) = \frac{s+1}{s-1} = 1 + \frac{2}{s-1} \]
		}
		%
		\item $ T_{y_2 u_1}(s) $:
		\[ T_{y_2 u_1}(s) = \frac{\frac{s+1}{s-1}\frac{1}{s+1}}{1+\frac{s^2}{(s+1)(s-1)}} = \frac{1}{s^2} \]
		%
		\item $ T_{y_2 u_2}(s) $:
		\[ T_{y_2 u_2}(s) = \frac{\frac{1}{s-1}}{\frac{s^2}{(s+1)(s-1)}} = \frac{s+1}{s^2}\]
	\end{itemize}
	%
	\section{Esercizio}
	Calcolare la funzione di trasferimento del sistema:\\
	\begin{figure}[h!]
		\centering
		\includegraphics[width=\textwidth]{interconnections/algebra_blocchi_es2_1}
	\end{figure}

	Spostiamo il nodo $ e $ e otteniamo un nuovo sistema composto da una retroazione e da un parallelo.
	L'algebra dei blocchi permette di fare queste equivalenze solo per il calcolo della funzione di trasferimento complessiva e non per altri calcoli, perch\'e stiamo introducendo nuovi blocchi, tra cui alcuni irrealizzabili (vedi il blocco di funzione di trasferimento $ s $).
	\begin{figure}[h!]
		\centering
		\begin{subfigure}{0.5\textwidth}
			\includegraphics[width=\textwidth]{interconnections/algebra_blocchi_es2_2}
		\end{subfigure}%
		\begin{subfigure}{0.5\textwidth}
			\includegraphics[width=\textwidth]{interconnections/algebra_blocchi_es2_3}
		\end{subfigure}%
	\end{figure}
	\[ T_{yu}(s) = \frac{\frac{1}{s}}{1+\frac{1}{s}} \left( s+\frac{1}{s} \right) = \frac{s^2+1}{s(s+1)} \]
\end{document}
\end{document}