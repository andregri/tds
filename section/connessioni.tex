\documentclass[../main.tex]{subfiles}

\begin{document}
	\section{Connessioni}
	Le possibili connessioni tra i sistemi sono:
	\begin{figure}[h!]
		\centering
		\begin{subfigure}{0.5\textwidth}
			\includegraphics[width=\textwidth]{interconnections/sum}
			\caption{somma}
		\end{subfigure}%
		\begin{subfigure}{0.5\textwidth}
			\includegraphics[width=\textwidth]{interconnections/diramation}
			\caption{diramazione}
		\end{subfigure}
	\end{figure}
	%
	\section{Connessioni equivalenti}
	Sono utili \textbf{solo} per trovare la funzione di trasferimento totale di un sistema nell'algebra dei blocchi. Non possono essere usate per studiare complessivamente un sistema, perch\'e non tengono conto delle condizioni iniziali e quindi delle risposte libere dei sistemi.
	\begin{figure}[h!]
		\includegraphics[width=0.8\textwidth]{interconnections/connessione_eq1}
		\caption{Spostamento diramazione a monte di un blocco}
	\end{figure}
	\begin{figure}[h!]
		\includegraphics[width=0.8\textwidth]{interconnections/connessione_eq2}
		\caption{Spostamento diramazione a valle di un blocco}
	\end{figure}
	\begin{figure}[h!]
		\includegraphics[width=0.8\textwidth]{interconnections/connessione_eq3}
		\caption{Spostamento nodo sommatore a valle di un blocco}
	\end{figure}
	\begin{figure}[h!]
		\includegraphics[width=0.8\textwidth]{interconnections/connessione_eq4}
		\caption{Spostamento nodo sommatore a monte di un blocco}
	\end{figure}
	\begin{figure}[h!]
		\includegraphics[width=0.8\textwidth]{interconnections/connessione_eq5}
		\caption{Accumulo e scambio di nodi sommatori}
	\end{figure}
	\begin{figure}[h!]
		\includegraphics[width=0.8\textwidth]{interconnections/connessione_eq6}
		\caption{Accumulo e scambio di nodi di diramazione}
	\end{figure}
\end{document}