\documentclass[../main.tex]{subfiles}

\begin{document}
	\section{Applicazione dell'anti-trasfromata di Laplace a equazioni differenziali lineari a coefficienti costanti}
	Consideriamo una generica equazione differenziale lineare a coefficienti costanti:
	\begin{equation} \label{eq_diff_lin}
		y^{(n)}(t) + a_{n-1}y^{(n-1)}(t) + \dots + a_{(1)}y^{1}(t) + a_{0}y(t) = b_{m}u^{(m)}(t) + \dots + b_0 u(t)
	\end{equation}
	Dobbiamo trovare la soluzione $ y(t) $ che soddisfi questa equazione dato l'ingresso $ u(t) $ e le eventuali condizioni al contorno.\\
	\medskip\\
	Chiamiamo $ D $ l'operatore derivata rispetto al tempo:
	\[ D = \der{}{t} \quad \Rightarrow \quad y^{(k)}(t) = D^k y(t) \]
	\[ A(D) = D^n + a_{n-1} D^{n-1} + \dots + a_1 D + a_0 \]
	\[ B(D) = b_m D^m + b_{m-1} D_{m-1} + \dots + b_1 D + b_0 \]
	Allora l'equazione \ref{eq_diff_lin} diventa:
	\[ A(D) y(t) = B(D) u(t) \]
	Applichiamo la trasformata di Laplace:
	\[ \Lbrace{y^{(n)}(t) + a_{n-1}y^{(n-1)}(t) + \dots + a_{1}y^{1}(t) + a_{0}y(t)} = \Lbrace{b_{m}u^{(m)}(t) + \dots + b_0 u(t)} \]
	Sfrutto la propriet\'{a} di linearit\'{a} e applico la trasformata ad ogni singolo termine.\\
	Inoltre chiamo $ \Lbrace{y(t)} = Y(s) $ e $ \Lbrace{u(t)} = U(s) $.
	\allowdisplaybreaks
	\begin{align*}
		\Lapl{y^{(n)}(t)&}{s^n Y(s) &&- \left[ s^{n-1} y(0^-) + \dots + y^{(n-1)}(0^-) \right] }\\
		\Lapl{+ a_{n-1}y^{(n-1)}(t)&}{a_{n-1} s^{n-1} Y(s) &&- a_{n-1} \left[ s^{n-2} y(0^-) + \dots + y^{(n-2)}(0^-) \right] }\\
		\smallskip\dots\smallskip\\
		\Lapl{+ a_{1}y^{(1)}(t)&}{a_{1} s Y(s) &&- a_{1} y(0^-) }\\
		\Lapl{+ a_{0}y(t)&}{a_{0} Y(s) } = \\\medskip\\
		\Lapl{ = b_m u^{(m)}(t)& }{b_m s^m U(s) &&- b_m \left[ s^{m-1} u(0^-) + \dots + u^{(m-1)}(0^-) \right] }\\
		\smallskip\dots\smallskip\\
		\Lapl{+ b_{1}y^{(1)}(t)&}{b_{1} s U(s) &&- b_{1} u(0^-) }\\
		\Lapl{+ b_{0}u(t)&}{b_{0} U(s) }
	\end{align*}
	I coefficienti di $ Y(s) $ costituiscono proprio il polinomio $ A $ calcolato in $ s $, mentre i coefficienti di $ U(s) $ sono il polinomio $ B $ calcolato in $ s $. Se raccolgo $ Y(s) $ e $ U(s) $:
	\[ A(s) Y(s) - I_y(s) = B(s) U(s) - I_u(s) \]
	dove:
	\begin{itemize}
		\item $ I_y(s) $ \'{e} un polinomio in $ s $ con grado massimo pari a $ n-1 $ e coefficienti dipendenti dalle condizioni iniziali $ y(0^-)\ \dots\ y^{(n-1)}(0^-) $
		\item $ I_u(s) $ \'{e} un polinomio in $ s $ con grado massimo pari a $ m-1 $ e coefficienti dipendenti dalle condizioni iniziali $ u(0^-)\ \dots\ u^{(m-1)}(0^-) $
	\end{itemize}
	la soluzione dell'equazione differenziale nel dominio di Laplace \'{e}:
	\begin{equation}
		Y(s) = \frac{B(s)}{B(s)} U(s) + \frac{I(s)}{A(s)} \qquad \text{con}\: I(s) = I_y(s) - I_u(s) 
	\end{equation}
	Quindi per trovare la soluzione nel domini del tempo basta anti-trasfromare:
	\[ y(t) = \aLbrace{Y(s)} \]
	\smallskip\\
	Riassumendo, i dati che mi servono per calcolare $ Y(s) $ sono:
	\begin{itemize}
		\item l'ingresso: $ u(t) \quad t \in [0, +\infty) $
		\item le condizioni iniziali: $ y(0^-)\ \dots\ y^{(n-1)}(0^-) \qquad u(0^-)\ \dots\ u^{(m-1)}(0^-) $ 
	\end{itemize}
	Apparentemente mi servono $ n+m $ condizioni iniziali.\\
	Ma $ I(s) $ \'{e} un polinomio di grado massimo pari a $ \max(n,m) - 1 $, quindi costituito da $ \max(n,m) $ coefficienti (perch\'{e} si esclude il coefficiente del termine di grado zero).\\
	Ne segue che ci sono $ \max(n,m) $ condizioni iniziali. Il modo in cui i coefficienti di $ I(s) $ dipendano da queste ultime \'{e} fissato dall'equazione differenziale.
	\smallskip\\
	Il numero di coefficienti di $ I(s) $ determina i gradi di libert\'{a} di $ \frac{I(s)}{A(s)} $.
	\[ y(t) = \aLbrace{ \frac{B(s)}{A(s)} U(s) } + \aLbrace{ \frac{I(s)}{A(s)} }\]
	\begin{align*}
		y_f(t) &= \aLbrace{ \frac{B(s)}{A(s)} U(s) } \quad &&\text{\textbf{risposta forzata} (risposta a condizioni iniziali nulle)}\\
		y_l(t) &= \aLbrace{ \frac{I(s)}{A(s)} } \quad &&\text{\textbf{risposta libera} (risposta a ingresso nullo)}
	\end{align*}
	\[ Y(s) = \frac{B(s)}{A(s)} U(s) + \frac{I(s)}{A(s)} = Y_f(s) + Y_l(s) \]	
	%
	\subsection{Funzione di trasferimento}
	\[ T(s) = \frac{\bar{A}(s)}{\bar{B}(s)} = \frac{\bar{A}(s)\ P(s)}{\bar{B}(s)\ P(s)} \]
	Se fattorizzando $ A(s) $ e $ B(s) $ troviamo dei fattori comuni $ P(s) $, cio\`{e}:
	\[ A(s) = \bar{A}(s)\ P(s) \qquad B(s) = \bar{B}(s)\ P(s) \]
	$ \bar{A}(s) $ e $ \bar{B}(s) $ sono detti \underline{coprimi} perch\`{e} non hanno alcuno fattore in comune.\\
	Quindi $ T(s) $ \'{e} funzione di trasferimento \textbf{se e solo se} corrisponde al rapporto tra $ B(s) $ e $ A(s) $ dopo aver cancellato tutti i fattori comuni.
	\subsection{Risposta all'impulso}
	\begin{itemize}
		\item le condizioni iniziali sono nulle $ (\Leftrightarrow I(s) = 0 ) $ e quindi anche $ Y_l(s) = 0 $
		\item $ u(t) = \delta(t) \quad \Rightarrow \quad U(s) = 1 $
	\end{itemize}
	\begin{align*}
		Y_f(s) &= T(s)U(s) = T(s)\\
		y_f(t) &= \aLbrace{T(s)} = h(t) \qquad \text{risposta forzata all'impulso}
	\end{align*}
	Quindi otteniamo anche:
	\[ \Lapl{Y_f(s) = T(s)U(s)}{y_f(t) = h(t) * u(t)} \]
	%
	\subsection{Esempio circuito RC}
	L'equazione del circuito \'{e}:
	\[ \dot{y}(t) + \frac{1}{RC} y(t) = \frac{1}{RC} u(t)  \]
	Introduciamo l'operatore derivata:
	\[ (D + \frac{1}{RC}) y(t) = \frac{1}{RC} u(t) \]
	\[ A(D) = D + \frac{1}{RC} \qquad \qquad B(D) = \frac{1}{RC} \]
	\[ A(s) = s + \frac{1}{RC} \qquad \qquad B(s) = \frac{1}{RC} \]
	La funzione di trasferimento \'{e}:
	\[ T(s) = \frac{B(s)}{A(s)} = \frac{\frac{1}{RC}}{s+\frac{1}{RC}}\]
	Trasformiamo l'equazione differenziale per ricavare $ I(s) $:
	\[ s Y(s) - y(0^-) + \frac{1}{RC} Y(s) = \frac{1}{RC} U(s) \]
	\[ Y(s) \underbrace{ \left( s + \frac{1}{RC} \right) }_{A(s)} - \underbrace{y(0^-)}_{I_y(s)} = \underbrace{\frac{1}{RC}}_{B(s)} U(s) - \underbrace{0}_{I_u(s)} \]
	Quindi abbiamo trovato che:
	\begin{itemize}
		\item $ I_y(s) $ \'e un polinomio di primo grado perch\'e $ y(t) $ compare derivato nell'equazione differenziale $ (n = 1) $.
		\item $ I_y(s) $ \'e un polinomio di grado zero perch\'e $ u(t) $ non \'e derivato $ (m = 1) $. 
	\end{itemize}
	Calcoliamo la risposta libera e la risposta forzata del sistema:
	\begin{itemize}
		\item $ u(t) = 1(t) \quad \text{e condizioni iniziali nulle} $\\
		\[ U(s) = \frac{1}{s} \quad \Rightarrow \quad Y(s) = Y_f(s) = \frac{\frac{1}{RC}}{s(s+\frac{1}{RC})} \]
		\[ y(t) = y_f(t) = \aLbrace{\frac{\frac{1}{RC}}{s(s+\frac{1}{RC})}} \]
		Si pu\'{o} calcolare attraverso i fratti semplici:
		\[ \frac{\frac{1}{RC}}{s(s+\frac{1}{RC})} = A\frac{1}{s} + B\frac{1}{s+\frac{1}{RC}} = \frac{(A+B)s+\frac{A}{RC}}{s(s+\frac{1}{RC})} \]
		\[ \begin{cases} A+B=1\\ A=1 \end{cases} \quad \begin{cases} A=1\\ B=-1 \end{cases} \]
		\[ y_f(t) = (1-e^{\frac{t}{RC}}) \cdot 1(t) \]
		%
		\item $ u(t) = 0 \quad \text{e condizioni iniziali non nulle} $\\
		\[ Y_l(s) = \frac{I(s)}{A(s)} = \frac{y_0}{s+\frac{1}{RC}} \]
		\[ y_l(t) = y_0\ e^{-\frac{t}{RC}} \cdot 1(t) \]
	\end{itemize}
	\medskip
	Poich\'{e} il sistema \'{e} lineare si pu\'{o} applicare il principio di sovrapposizione degli effetti: somma di cause corrisponde una somma di effetti. Quindi se l'ingresso e le condizioni iniziali non sono nulle, in uscita si avr\'{a} la somma dell'uscita forzata e di quella libera:
	\[ Y(s) = Y_f(s) + Y_l(s) = \frac{\frac{1}{RC}}{s+\frac{1}{RC}}U(s) + \frac{y_0}{s+\frac{1}{RC}} \]
	Inoltre \'{e} possibile applicare il teorema del valore finale perch\'{e} i poli sono a $ \Re = 0\ e\ \Re < 0 $
	\[ y_\infty = \lim\limits_{s \rightarrow 0} sY(s) = \lim\limits_{s \rightarrow 0} \frac{sy_0+\frac{1}{RC}}{s+\frac{1}{RC}} = \frac{\frac{1}{RC}}{\frac{1}{RC}} = 1 \]
\end{document}